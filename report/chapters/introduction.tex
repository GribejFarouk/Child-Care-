% ============================================================================
% Introduction Générale
% ============================================================================
\chapter*{Introduction Générale}
\addcontentsline{toc}{chapter}{Introduction Générale}
\markboth{Introduction Générale}{Introduction Générale}

Le suivi de la santé infantile constitue un enjeu majeur pour les familles et les professionnels de la santé. Dès les premières semaines de vie, la surveillance régulière de la croissance --- poids, taille, périmètre crânien --- ainsi que le respect du calendrier vaccinal et la détection précoce d'éventuelles anomalies sont essentiels au bon développement de l'enfant. En Tunisie, ce suivi repose encore largement sur des carnets de santé papier, des visites périodiques chez le pédiatre et des pratiques manuelles souvent sujettes à l'oubli ou à la perte d'information.

Dans un contexte de transformation numérique croissante, les applications mobiles et web de santé se multiplient à l'échelle mondiale. Toutefois, les solutions existantes présentent des limitations notables : interfaces peu adaptées au contexte tunisien, absence de référence aux normes de l'Organisation Mondiale de la Santé (OMS), fragmentation des données entre plusieurs outils, ou encore manque de fonctionnalités d'alerte intelligente.

C'est dans ce cadre que s'inscrit le projet \textbf{ChildCare+}, une plateforme numérique de suivi de la santé infantile. Ce projet de fin d'études, réalisé dans le cadre de la Licence en Génie Logiciel et Système d'Information à l'Université de Monastir, vise à concevoir et développer une application web complète permettant aux parents de :
\begin{itemize}
    \item Enregistrer et consulter les données de croissance de leurs enfants ;
    \item Visualiser l'évolution sur des courbes de croissance conformes aux normes OMS ;
    \item Recevoir des alertes intelligentes en cas de déviation par rapport aux seuils normaux ;
    \item Numériser des documents médicaux grâce à la reconnaissance optique de caractères (OCR) ;
    \item Suivre une grossesse en cours avec des jalons et des rappels personnalisés.
\end{itemize}

L'application adopte une architecture orientée microservices afin de garantir la scalabilité, la maintenabilité et la modularité du système. Le développement suit la méthodologie agile Scrum, avec une démarche incrémentale organisée en sprints.

\vspace{0.5cm}
Le présent rapport est structuré en quatre chapitres :

\begin{itemize}
    \item \textbf{Chapitre 1 --- Cadre général du projet :} présente le contexte, la problématique, les objectifs du projet, l'étude de l'existant et la méthodologie adoptée.
    \item \textbf{Chapitre 2 --- Analyse et spécification des besoins :} identifie les acteurs, les besoins fonctionnels et non fonctionnels, et formalise les cas d'utilisation.
    \item \textbf{Chapitre 3 --- Conception :} détaille l'architecture logicielle, les modèles de données et les diagrammes de séquence.
    \item \textbf{Chapitre 4 --- Réalisation :} présente l'environnement technique, les interfaces développées et l'état d'avancement actuel du projet.
\end{itemize}
